\documentclass[12pt, a4paper]{article}
\usepackage{graphicx} % Required for inserting images
\graphicspath{{./images}}
\usepackage{pgfplots}
\pgfplotsset{compat=1.18}
\usepackage{mathtools}
\usepackage{fancyhdr}
\usepackage{cancel}
\usepackage[top=1in, left = 1in, right = 1in, bottom=1.2in]{geometry}
\usepackage{listings}
\usepackage{booktabs}
\usepackage{tabularx}
\usepackage{subfig}
\usepackage{hyperref}
\usepackage{float}

% Herlink setup
\hypersetup{
    colorlinks=true,
    linkcolor=blue,
    filecolor=magenta,      
    urlcolor=cyan,
    pdftitle={Overleaf Example},
    pdfpagemode=FullScreen,
}

% For the code blocks
\definecolor{codegreen}{rgb}{0.03,0.5,0.03}
\definecolor{codegray}{rgb}{0.5,0.5,0.5}
\definecolor{codepurple}{rgb}{0.58,0,0.82}
\definecolor{backcolour}{rgb}{0.95,0.95,0.95}

% Code block setup
\lstdefinestyle{mystyle}{
    backgroundcolor=\color{backcolour},
    commentstyle=\color{codegreen},
    keywordstyle=\color{magenta},
    numberstyle=\tiny\color{codegray},
    stringstyle=\color{codepurple},
    basicstyle=\ttfamily\footnotesize,
    breakatwhitespace=false,
    breaklines=true,
    captionpos=b,
    keepspaces=true,
    numbers=left,
    numbersep=5pt,
    showspaces=false,
    showstringspaces=false,
    showtabs=false,
    tabsize=2,
    escapeinside = {(*}{*)}
}
\lstset{style=mystyle}

% My custom headers and margins 
\pagestyle{fancy}
\setlength{\headheight}{44pt}
\setlength{\headsep}{18pt}
\lhead{\includegraphics[scale = 0.2]{~/Documents/masters/bnw unit.png}}
\chead{\quad Data Science and Information Technologies Master’s
National and Kapodistrian University of Athens}
\rhead{}
\lfoot{}
\cfoot{\thepage}
\rfoot{}

% Start
\begin{document}

% Custom title page
\begin{titlepage}
    \centering
    {\huge \textbf{Homework}\par}
    \vspace{0.5cm}
    {\Large \textbf{Name:} Aris Podotas\par}
    \vspace{0.5cm}
    {\large \textbf{University:} National and Kapodistrian University of Athens\par}
    \vspace{0.5cm}
    {\large \textbf{Program:} Data Science and Informaion Technologies\par}
    \vspace{0.5cm}
    {\large \textbf{Specialization:} Bioinformatics - Biomedical Data\par}
    \vspace{0.5cm}
    {\large \textbf{Lesson:} Clustering Algorithms \par}
    \vspace{0.5cm}
    {\large \textbf{Date:} November 2024\par}
    \tableofcontents
\end{titlepage}

\section{Feeling the data}

Before any of the further analysis a look at the files provided in the report are all the files that were originally sent with the description along with a file named $main.m$ that implements the Matlab code that generates all the solutions and a file named $holder.pdf$ that contains this report. It is advised that one looks at the $main.m$ file before runnning it and commenting out segments that are time consuming when trying to consider just one part of the output.

\subsection{Missing data}

It is stated in the description of the project that ().
\newline

Identifying if missing data exists and where it exists if it does has been done with the following matlab code:
\begin{lstlisting}[language=Matlab, label=lst:missing, caption=Identifying missing data.]
% Loading the data
\end{lstlisting}

How will we handle missing data?

\subsection{Data type}

This segment is about the values our data takes (discreet or continous). There is no good way to represent this in matlab code, so we will take a look at the loeaded variable, the values it takes for all featrues and the features themselves.
\newline

Input data:

\begin{figure}
    \begin{center}
        \includegraphics[width=0.95\textwidth]{figures/pending.png}
    \end{center}
    \caption{}\label{fig:data}
\end{figure}

From Figure \ref{fig:data} we can see that our features are ().

\section{Feature selection/transformation}

\section{Selection of the clustering algorithm}

\section{Execution of the algorithms}

\section{Characterization of the clusters}

\section{Citations}

\end{document}
